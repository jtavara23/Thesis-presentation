%!TEX root = start.TEX
\frame{\titlepage}

\AtBeginSection[]{
\begin{frame}
\frametitle{
	\begin{center}
	\vskip -0.8cm
	{\fontsize{13}{13.4}\selectfont{Modelo De Reconocimiento Automático De Señales De Tránsito Vehicular Mediante Aprendizaje Profundo De Redes Neuronales Convolucionales}}
	\end{center}
	\vskip -1cm
	}
	\tableofcontents[currentsection]
	\end{frame}
}

\frame{
\begin{abstract}
\justifying
%\begin{center}
 La presente investigación tiene por objetivo principal implementar un modelo basado en el aprendizaje profundo de redes neuronales convolucionales para reconocer automáticamente señales de tránsito vehicular usando fundamentos de  técnicas de procesamiento de imágenes y algoritmos de inteligencia artificial.
  \vskip 0.2cm
  El proyecto se centra en un grupo de señales de Tránsito vehicular de Alemania y Perú, identificando 43 y 7 categorías respectivamente. Iniciando con  la adquisición de imágenes, se procedió realizar el procesamiento de estas con la finalidad de aumentar el conjunto de datos y poder ejecutar el aprendizaje profundo a través de diversos diseños de arquitecturas de redes neuronales convolucionales.
  \vskip 0.2cm
  Como resultado final, se obtuvo un modelo con buenos indicadores y resultados en el reconocimiento de señales de tránsito vehicular. De esta manera, se pretende contribuir en los esfuerzos de la industria automotriz en el campo de sistemas avanzados de asistencia al conductor así como también puede formar parte de diversos mecanismos que buscan dar soluciones a la inseguridad vial.
  \vskip 0.2cm
 	\begin{center} 
  {\bf Palabras claves:} aprendizaje profundo, redes neuronales convolucionales, procesamiento de imágenes.
	\end{center}
\end{abstract}
}

%-------------------------------------------------------------------------------------------------
%-------------------------------------------------------------------------------------------------
\section{Introducción}
	\frame{
	\begin{block}
	{\Large{Introducción}}
	\end{block}
	\vskip 0.5cm
	\begin{itemize}
		
		\item<1->Al conducir en pistas o carreteras, a veces es difícil mantener los ojos en todas partes a la vez, comprobando el camino por delante, hacia donde girar, dónde disminuir la velocidad o tratar de mantenerla y no acelerar; es por ello, que existen mecanismos destinados a reglamentar el tránsito, advertir o informar a los usuarios mediante palabras, sonidos o símbolos determinados.

		\item<2->La policía de tránsito o las {\bf señalizaciones vehiculares} regulan el tránsito e informan al usuario sobre direcciones, rutas, advertencias, así como dificultades existentes en las carreteras y previenen cualquier peligro o infracción que podría presentarse durante la circulación vehicular.

	\end{itemize}
	}

	\frame{
	\begin{block}
	{\Large{Introducción}}
	\end{block}
	\vskip 0.5cm
	\begin{itemize}

		\item<1->Sin embargo, cuando estos mecanismos no son reconocidos o percibidos pueden ocasionar no solo que la congestión del tráfico aumente, sino que también se produzcan accidentes que en muchos casos derivan en consecuencias fatales, generando inseguridad vial.
		
		\item<2-> La inseguridad vial es un problema de interés mundial, según el último informe de la OMS (Organización Mundial de la salud) anualmente cerca de 1,3 millones de personas mueren alredor del mundo y entre 20 y 50 millones padecen traumatismos no mortales,\citep{OMS}. Son distintas las causas que conllevan a este problema, de las cuales las principales pueden ser la falta de concientización y educación vial.
	\end{itemize}
	}

	\frame{
	\begin{block}
	{\Large{Introducción}}
	\end{block}
	\vskip 0.5cm
	\begin{itemize}
		 
		\item<1-> Es por ello que trabajar en obtener vehículos más seguros es un factor fundamental para prevenir de alguna forma los accidentes de tránsito o reducir la probabilidad de que estos sean producidos, \citep{OMS}.
		
	\end{itemize}
	}

%-------------------------------------------------------------------------------------------------
%-------------------------------------------------------------------------------------------------

\section{Motivación}
	\frame{
	\begin{block}
	{\Large{Motivación}}
	\end{block}
	\vskip 0.2cm
	\begin{itemize}
		\item<1-> Para contribuir con lo antes mencionado, se han venido planteando formas que permitan la asistencia en el reconocimiento de señales de tránsito, la cual es un problema de clasificación que comúnmente presenta desigualdades en las frecuencias de aparición de las categorías. Además, las señales de tránsito muestran una amplia gama de variaciones entre las clases en términos de color, forma, subconjuntos de clases que son muy similares entre sí y la presencia de símbolos, leyendas o texto. A esto es sumado, las grandes variaciones en las apariencias visuales debido a cambios de iluminación, oclusiones parciales, rotaciones, condiciones meteorológicas, escalamiento, etc. 

		\item<2-> Todo esto representa un reto para el reconocedor/clasificador de señales de tránsito vehicular y es por ello que se han venido realizando diversas investigaciones.%, donde esta forma parte de una de ellas.
	\end{itemize}
	}
%-------------------------------------------------------------------------------------------------
%-------------------------------------------------------------------------------------------------
\section{Formulación del problema e Hipótesis}
	\frame{
	\begin{block}
	{\Large{Formulación del problema}}
	\end{block}
	\vskip 0.3cm
	 En este trabajo, se propone responder a la siguiente pregunta:
	 \vskip 0.3cm
	 \begin{center} 
	     \textbf{¿Cómo se puede reconocer de manera automática señales de tránsito vehicular?}
	 \end{center}

	 \begin{block}
	{\Large{Hipótesis}}
	\end{block}
	 \vskip 0.3cm
	 \begin{center} 
	    Un modelo basado en el aprendizaje profundo de redes neuronales convolucionales permitirá el
reconocimiento automático de señales de tránsito vehicular.
	 \end{center}

	}

%-------------------------------------------------------------------------------------------------
%-------------------------------------------------------------------------------------------------

\section{Importancia de la investigación} 
	\frame{
	\vskip -1cm
	\begin{block}
	{\Large{Importancia de la investigación - Justificación Académica}}
	\end{block}
	\vskip 0.3cm
	\begin{itemize}
	\item<1->  La importancia de esta investigación en el punto de vista de ciencias de la computación se justifica en poner en práctica los conocimientos adquiridos en la formación académica, siendo los más resaltables el tema de procesamiento de imágene e inteligencia artificial, con la finalidad de obtener un modelo robusto de redes neuronales convolucionales basadas en el aprendizaje profundo (deep learning) que permita el reconocimiento de señales de tránsito vehicular. 
	\vskip 0.15cm
	\item<2->  Con los rápidos avances de las estructuras de algoritmos de aprendizaje profundo y la factibilidad de su implementación de alto rendimiento con unidades de procesamiento gráfico (GPU), es ventajoso investigar en problemas de clasificación de imágenes desde la perspectiva de un aprendizaje profundo eficiente, \citep{recentCNN}. Siendo está la primera investigación realizada en base a imágenes de señales de tránsito vehicular del Perú.
	\end{itemize}
	}

	\frame{
	\vskip -0.5cm
	\begin{block}
	{\Large{Importancia de la investigación - Justificación Social}}
	\end{block}
	\vskip 0.2cm
	\begin{itemize}
	\item<1->  Teniendo conocimiento de lo descrito sobre la realidad problemática, la visibilidad y conocimiento de señales de tráfico es crucial para la seguridad de los conductores y es por ello que la introducción de un modelo de reconocimiento de señales de tránsito que funcione en diferentes contextos puede formar parte de la solución a que constantes infracciones y en consecuencia evitar o reducir estos indices progresivamente.  
	\vskip 0.1cm
	\item<2-> Algunos ejemplos de apliaciones inmediatas:
	\vskip 0.1cm
	\begin{enumerate}	
		\item[--]<3-> Dar una notificación de ciertas restricciones en el límite de velocidad.
		 \vskip 0.1cm
		\item[--]<4-> Recibir un aviso de no estacionarse para posteriormente evitar un infracción
		\vskip 0.1cm
		\item[--]<5-> Darse cuenta de que está cometiendo una infracción al girar hacia la derecha al recibir un aviso de que se debe girar solo hacia la izquierda
		\vskip 0.1cm
		\item[--]<6-> Una aplicación móvil que dé la posibilidad de reconocer automáticamente aquella señal de tránsito que usuarios puedan desconocer serviría como aporte en la educación vial. 	
	\end{enumerate} 
	\end{itemize}
	}

	\frame{
	\vskip 0.5cm
	\begin{block}
	{\Large{Importancia de la investigación - Justificación Social}}
	\end{block}
	\vskip 0.2cm
	\begin{itemize}
	\item<1-> Es importante porque a través de un modelo que reconozca señales de tránsito vehicular se podría contribuir en la industria automotriz, específicamente en los sistemas avanzados de asistencia al conductor (del inglés, ADAS); así como también se ha descrito anteriormente, el modelo pretendido puede ser usado para formar parte de diversos mecanismos que buscan dar soluciones a la inseguridad vial. 
	
	\end{itemize}
	}

%-------------------------------------------------------------------------------------------------
%-------------------------------------------------------------------------------------------------



\section{Contribución de la investigación}
\frame{
\begin{block}
{\Large{Contribución de la investigación}}
\end{block}
\vskip 0.5cm
\begin{itemize}
\item<1->Todos los hechos descritos hacen que el reconocimiento de las señales de tránsito sea un reto desafiante y esencial en muchos aspectos, no solo para contribuir en los esfuerzos de la industria automotriz en el campo de la asistencia al conductor, sino también para organismos internacionales y gubernamentales  que buscan constantemente introducir nuevos mecanismos y tecnologías que faciliten y mejoren la conducción vehicular. Es por ello que esta investigación contribuye de la siguiente manera:
\end{itemize}
}




\frame{
\begin{block}
{\Large{Contribución de la investigación}}
\end{block}
\vskip 0.5cm
\begin{itemize}
\item<1-> Otorga un modelo computacional para el reconocimiento de 43 tipos de señales de Tránsito de Alemania, con una tasa de acierto del {\bf 98.62\%}, mucho mejor que el resultado de 95.29\% obtenido por \cite{Ayuque2016} y mucho más próximo al mejor resultado de 99.46\% obtenido en las investigaciones hechas por \cite{Ciresan} en base al dataset GTSRB.
\vskip 0.5cm
\item<2-> Otorga un modelo computacional para el reconocimiento de 7 tipos señales de Tránsito del Perú, el cual posee un {\bf(99.02\%)} de acierto tras analizar una muestra de 4698 imágenes. 
\vskip 0.5cm
\item<3-> La investigación ofrece para futuras investigaciones, un dataset de señales de Tránsito del Perú compuesto por 31314 imágenes distribuidas en 7 categorías.
\end{itemize}
}

%-------------------------------------------------------------------------------------------------
%-------------------------------------------------------------------------------------------------


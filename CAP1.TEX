%!TEX root = origin.TEX
\frame{\titlepage}

\AtBeginSection[]{
\begin{frame}
\frametitle{
\begin{center}
{Modelo De Reconocimiento Automático De Señales De Tránsito Vehicular Mediante Aprendizaje Profundo De Redes Neuronales Convolucionales}
\end{center}
}
\tableofcontents[currentsection]
\end{frame}
}

\frame{
\begin{abstract}
Esta investigación pretende contribuir en la industria automotriz, especificamente en los campos de construcción de vehículos autónomos y de los sistemas avanzados de asistencia al conductor, iniciando con la adquisición de imágenes, luego para el pre-procesamiento, se implementará algoritmos de realce de contraste, reducción de ruido, rotación y proyecciones de escalamiento con la finalidad de aumentar el conjunto de datos y poder ejecutar el aprendizaje profundo a través de arquitecturas de redes neuronales convolucionales. Se van a realizar diferentes diseños de arquitecturas convolucionales y se escogerá el que obtenga los mejores resultados.
\hspace*{0.0cm}{\bf Palabras claves:} aprendizaje profundo, redes neuronales convolucionales, procesamiento de imágenes.
\end{abstract}
}

%-------------------------------------------------------------------------------------------------
%-------------------------------------------------------------------------------------------------
\section{Introducción}
\frame{
\begin{block}
{\Large{Introducción}}
\end{block}
\vskip 0.5cm
\begin{itemize}
\item<1-> A medida 
\vskip 0.5cm
\item<2-> Las   
\end{itemize}
}


\section{Motivación}
\frame{
\begin{block}
{\Large{Motivación}}
\end{block}
\vskip 0.5cm
\begin{itemize}
\item<1-> De esta forma,
\vskip 0.5cm
\item<2-> Por lo tanto, 
\end{itemize}
}

\section{Formulación del problema}
\frame{
\begin{block}
{\Large{Formulación del problema}}
\end{block}
\vskip 0.5cm
  En este trabajo, se propone discutir el modelo de red  de logística reversa basado en el problema del ruteo de vehículos para responder a la siguiente pregunta:
\vskip 1cm  
 \begin{center} 
     ?`Cómo viabilizar una red logística reversa en regiones urbanas minimizando los costos logísticos de ruteo y transporte de los RSU hasta su disposición final?
 \end{center}
}


\section{Importancia de la investigación} 
\frame{
\begin{block}
{\Large{Importancia de la investigación}}
\end{block}
\vskip 0.5cm
\begin{itemize}
\item<1-> La  
\vskip 0.3cm
\item<2-> El escenario ... en la tabla 1.1.
\end{itemize}
}




\frame{
\begin{table}[h!]
\centering
\caption{Estimativa en los programas de colecta selectiva formal (2008)}
\begin{tabular}{|c|c|c|c|}  \hline 
Residuos & Residuos colectados(t/día) & Residuos reciclados(t/año) \\ \hline 
Metales & 5 293 & 9 817 \\
Papeles & 23 997 & 3 827 \\ 
Plástico & 24 847 & 962 \\
Vidrio & 4 388 & 489  \\\hline
\end{tabular}
\begin{center}
{\small{Fuente: \cite{MMA}.}}
\end{center}
\end{table}
}


\section{Contribución de la investigación}
\frame{
\begin{block}
{\Large{Contribución de la investigación}}
\end{block}
\vskip 0.5cm
\begin{itemize}
\item<1-> La  
\vskip 0.3cm
\item<2-> El 
\end{itemize}
}

\section{Marco teórico}
\frame{
\begin{block}
{\Large{Marco teórico}}
\end{block}
\vskip 0.5cm
\begin{itemize}
\item<1-> {\bf Optimización combinatoria:}
\begin{itemize}
\item<2->
\vskip 0.3cm
\item<3->
\vskip 0.3cm
\item<4->
\vskip 0.3cm
\item<5->
\end{itemize}
\end{itemize}
}

\frame{
\begin{itemize}
\item<1-> {\bf Complejidad computacional:}
\begin{itemize}
\item<2->
\vskip 0.3cm
\item<3->
\vskip 0.3cm
\item<4->
\vskip 0.3cm
\item<5->
\end{itemize}
\end{itemize}
}

\frame{
\begin{itemize}
\item<1-> {\bf Metaheurísticas:}
\begin{itemize}
\item<2-> Algoritmos genéticos:
\vskip 0.3cm
\item<3-> Busca Tabú:
\vskip 0.3cm
\item<4-> Simulated annealing:
\vskip 0.3cm
\item<5-> Ant colony:
\end{itemize}
\end{itemize}
}



\section{Propuesta o tema central de la tesis}
\frame{
\begin{block}
{\Large{Propuesta o tema central de la tesis}}
\end{block}
\vskip 0.5cm
\begin{itemize}
\item<1-> 
\vskip 0.5cm
\item<2->
\vskip 0.5cm
\item<3->
\vskip 0.5cm
\item<4->
\vskip 0.5cm
\item<5->
\end{itemize}
}


\section{Resultados de la tesis}
\frame{
\begin{block}
{\Large{Resultados de la tesis}}
\end{block}
\vskip 0.5cm
Al culminar con la investigación se llegaron a resultados interesantes del punto de vista tanto teórico como computacional. ...
\begin{itemize}
\item<1-> {\bf Teóricos:}
\begin{itemize}
\item<2->
\vskip 0.3cm
\item<3->
\end{itemize}
\end{itemize}
}

\frame{
\begin{itemize}
\item<1-> {\bf Computacionales:}
\begin{itemize}
\item<2->
\vskip 0.3cm
\item<3->
\end{itemize}
\end{itemize}
}


\section{Consideraciones finales}
\frame{
\begin{block}
{\Large{Consideraciones finales}}
\end{block}
\vskip 0.5cm
\begin{itemize}
\item<1-> {\bf Conclusiones:}
\begin{itemize}
\item<2->
\vskip 0.3cm
\item<3->
\end{itemize}
\end{itemize}
}

\frame{
\begin{itemize}
\item<1-> {\bf Trabajos futuros:}
\begin{itemize}
\item<2->
\vskip 0.3cm
\item<3->
\end{itemize}
\end{itemize}
}

\section{Referencias bibliograficas}
\frame{
\begin{block}
{\Large{Referencias bibliograficas}}
\end{block}
\vskip 0.5cm
 \bibliography{Bibliografia} % Bibliografia formato APA
}

%!TEX root = origin.TEX
\frame{\titlepage}

\AtBeginSection[]{
\begin{frame}
\frametitle{
\begin{center}
{Modelo De Reconocimiento Automático De Señales De Tránsito Vehicular Mediante Aprendizaje Profundo De Redes Neuronales Convolucionales}
\end{center}
}
\tableofcontents[currentsection]
\end{frame}
}

\frame{
\begin{abstract}
\justifying
\begin{center}
Esta investigación pretende contribuir en la industria automotriz, especificamente en los campos de construcción de vehículos autónomos y de los sistemas avanzados de asistencia al conductor, iniciando con la adquisición de imágenes, luego para el pre-procesamiento, se implementará algoritmos de realce de contraste, reducción de ruido, rotación y proyecciones de escalamiento con la finalidad de aumentar el conjunto de datos y poder ejecutar el aprendizaje profundo a través de arquitecturas de redes neuronales convolucionales. Se realizaron diferentes diseños de arquitecturas convolucionales y se escogió el que obtuvo los mejores indicadores/resultados.
\vskip 0.8cm
\hspace*{0.0cm}{\bf Palabras claves:} aprendizaje profundo, redes neuronales convolucionales, procesamiento de imágenes.
\end{center}
\end{abstract}
}

%-------------------------------------------------------------------------------------------------
%-------------------------------------------------------------------------------------------------
\section{Introducción}
	\frame{
	\begin{block}
	{\Large{Introducción}}
	\end{block}
	\vskip 0.5cm
	\begin{itemize}
		
		\item<1->Al conducir en carreteras congestionadas, a veces es difícil mantener los ojos en todas partes a la vez, comprobando el camino por delante, el tráfico venidero, lo que está detrás de usted, tratar de mantener la velocidad permitida; es por ello, que existen mecanismos destinados a reglamentar el tránsito, advertir o informar a los usuarios mediante palabras, sonidos o símbolos determinados.

		\item<2->La policía de tránsito o las {\bf señalizaciones vehiculares} que según sea el caso, en todos los países regulan el tránsito e informan al usuario sobre direcciones, rutas, destinos, así como dificultades existentes en las carreteras y previenen cualquier peligro que podría presentarse en la circulación vehicular.

	\end{itemize}
	}

	\frame{
	\begin{block}
	{\Large{Introducción}}
	\end{block}
	\vskip 0.5cm
	\begin{itemize}
		
		\item<1-> La inseguridad vial es un problema de interés mundial, según el último informe de la OMS (Organización Mundial de la salud) anualmente cerca de 1,3 millones de personas mueren alredor del mundo y entre 20 y 50 millones padecen traumatismos no mortales,\citep{OMS}. Son distintas las causas que conllevan a este problema, de las cuales las principales pueden ser la falta de concientización y educación vial.
	\end{itemize}
	}

	\frame{
	\begin{block}
	{\Large{Introducción}}
	\end{block}
	\vskip 0.5cm
	\begin{itemize}
		
		\item<1-> El Perú y la capital Lima se encuentran respectivamente en la lista de peores países y ciudades para conducir en America Latina, según el Índice Global de Satisfacción del Conductor \citep{CNN}, lo cual se ve reflejado en que los últimos años se ha incrementado el índice de mortandad originados por los accidentes de tránsito siendo las principales causas de los mismos el exceso de velocidad, estado de ebriedad del conductor y sobretodo el desacato a las señales de tránsito, todas ellas de responsabilidad directa del conductor del vehículo motorizado,\citep{SUTRAN}. 
		 
		\item<2-> Es por ello que trabajar en obtener vehículos más seguros es un factor fundamental para prevenir de alguna forma los accidentes de tránsito o reducir la probabilidad de que estos sean producidos, \citep{OMS}.
		
	\end{itemize}
	}

%-------------------------------------------------------------------------------------------------
%-------------------------------------------------------------------------------------------------

\section{Motivación}
	\frame{
	\begin{block}
	{\Large{Motivación}}
	\end{block}
	\vskip 0.2cm
	\begin{itemize}
		\item<1-> Para contribuir con lo antes mencionado, se han venido planteando formas que permitan la asistencia en el reconocimiento de señales de tránsito, la cual es un problema de clasificación multicategórica que comúnmente presenta desigualdades en las frecuencias de aparición de las categorías. Además, las señales de tránsito muestran una amplia gama de variaciones entre las clases en términos de color, forma, subconjuntos de clases que son muy similares entre sí y la presencia de símbolos, leyendas o texto. A esto es sumado, las grandes variaciones en las apariencias visuales debido a cambios de iluminación, oclusiones parciales, rotaciones, condiciones meteorológicas, escalamiento, etc. 

		\item<2-> Todo esto representa un reto para el reconocedor/clasificador de señales de tránsito vehicular y es por ello que se han venido realizando diversas investigaciones, donde esta forma parte de una de ellas.
	\end{itemize}
	}
%-------------------------------------------------------------------------------------------------
%-------------------------------------------------------------------------------------------------
\section{Formulación del problema}
	\frame{
	\begin{block}
	{\Large{Formulación del problema}}
	\end{block}
	\vskip 0.3cm
	 En este trabajo, se propone discutir el modelo de redes neuronales convolucionales basado en el problema del reconocimiento de imágenes para responder a la siguiente pregunta:
	 \vskip 0.3cm
	 \begin{center} 
	     ¿Cómo se puede reconocer de manera automática señales de tránsito vehicular?
	 \end{center}
	}
%-------------------------------------------------------------------------------------------------
%-------------------------------------------------------------------------------------------------

\section{Importancia de la investigación} 
\frame{
\begin{block}
{\Large{Importancia de la investigación}}
\end{block}
\vskip 0.5cm
\begin{itemize}
\item<1-> La  
\vskip 0.3cm
\item<2-> El escenario ... en la tabla 1.1.
\end{itemize}
}

%-------------------------------------------------------------------------------------------------
%-------------------------------------------------------------------------------------------------


\frame{
\begin{table}[h!]
\centering
\caption{Estimativa en los programas de colecta selectiva formal (2008)}
\begin{tabular}{|c|c|c|c|}  \hline 
Residuos & Residuos colectados(t/día) & Residuos reciclados(t/año) \\ \hline 
Metales & 5 293 & 9 817 \\
Papeles & 23 997 & 3 827 \\ 
Plástico & 24 847 & 962 \\
Vidrio & 4 388 & 489  \\\hline
\end{tabular}
\begin{center}
{\small{Fuente: \cite{MMA}.}}
\end{center}
\end{table}
}


\section{Contribución de la investigación}
\frame{
\begin{block}
{\Large{Contribución de la investigación}}
\end{block}
\vskip 0.5cm
\begin{itemize}
\item<1-> La  
\vskip 0.3cm
\item<2-> El 
\end{itemize}
}

\section{Marco teórico}
\frame{
\begin{block}
{\Large{Marco teórico}}
\end{block}
\vskip 0.5cm
\begin{itemize}
\item<1-> {\bf Optimización combinatoria:}
\begin{itemize}
\item<2->
\vskip 0.3cm
\item<3->
\vskip 0.3cm
\item<4->
\vskip 0.3cm
\item<5->
\end{itemize}
\end{itemize}
}

\frame{
\begin{itemize}
\item<1-> {\bf Complejidad computacional:}
\begin{itemize}
\item<2->
\vskip 0.3cm
\item<3->
\vskip 0.3cm
\item<4->
\vskip 0.3cm
\item<5->
\end{itemize}
\end{itemize}
}

\frame{
\begin{itemize}
\item<1-> {\bf Metaheurísticas:}
\begin{itemize}
\item<2-> Algoritmos genéticos:
\vskip 0.3cm
\item<3-> Busca Tabú:
\vskip 0.3cm
\item<4-> Simulated annealing:
\vskip 0.3cm
\item<5-> Ant colony:
\end{itemize}
\end{itemize}
}



\section{Propuesta o tema central de la tesis}
\frame{
\begin{block}
{\Large{Propuesta o tema central de la tesis}}
\end{block}
\vskip 0.5cm
\begin{itemize}
\item<1-> 
\vskip 0.5cm
\item<2->
\vskip 0.5cm
\item<3->
\vskip 0.5cm
\item<4->
\vskip 0.5cm
\item<5->
\end{itemize}
}


\section{Resultados de la tesis}
\frame{
\begin{block}
{\Large{Resultados de la tesis}}
\end{block}
\vskip 0.5cm
Al culminar con la investigación se llegaron a resultados interesantes del punto de vista tanto teórico como computacional. ...
\begin{itemize}
\item<1-> {\bf Teóricos:}
\begin{itemize}
\item<2->
\vskip 0.3cm
\item<3->
\end{itemize}
\end{itemize}
}

\frame{
\begin{itemize}
\item<1-> {\bf Computacionales:}
\begin{itemize}
\item<2->
\vskip 0.3cm
\item<3->
\end{itemize}
\end{itemize}
}


\section{Consideraciones finales}
\frame{
\begin{block}
{\Large{Consideraciones finales}}
\end{block}
\vskip 0.5cm
\begin{itemize}
\item<1-> {\bf Conclusiones:}
\begin{itemize}
\item<2->
\vskip 0.3cm
\item<3->
\end{itemize}
\end{itemize}
}

\frame{
\begin{itemize}
\item<1-> {\bf Trabajos futuros:}
\begin{itemize}
\item<2->
\vskip 0.3cm
\item<3->
\end{itemize}
\end{itemize}
}

\section{Referencias bibliograficas}
\frame{
\begin{block}
{\Large{Referencias bibliograficas}}
\end{block}
\vskip 0.5cm
 \bibliography{Bibliografia} % Bibliografia formato APA
}
